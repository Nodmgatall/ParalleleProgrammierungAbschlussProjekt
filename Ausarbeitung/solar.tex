\title{Simulation eines Sonnensystems}
%TODO :
\subtitle{Praktikumsbericht}

\author{Oliver Heidmann, Tronje Krabbe}

\institute{Universität Hamburg \\ 
  Fakultät für Mathematik, Informatik und Naturwissenschaften \\
  Fachbereich Informatik, DKRZ \\
  Praktikum Paralleles Programmieren 2015}
  
\date{\today}

\maketitle

%\index{personen}{Krabbe, Tronje}
\newpage
\begin{abstract}
%TODO content
\quad \\
Aufgabenstellung: \\
Programmierung einer parallelisierten Applikation mittels MPI. \\ \\
Idee: \\
Simulation von Partikeln in einem Sonnensystem
%\keywords{Keywords}
\end{abstract}

\tableofcontents
\newpage
\section{Idee}
Lorem ipsum dolor sit amed

\section{Modellierung}
In diesem Abschnitt wird ein Überblick über die Modellierung der Ideen, der Welt und der Menschen in ihr gegeben. 
Details zur eigentlichen Implementation, insbesondere zur Aufteilung auf mehrere Prozesse, sind im Abschnitt Implementation zu finden.
\subsection{Das System}
\begin{minipage}[t]{0.48\textwidth}
    Dreidimensionaler raum mkay?
\end{minipage}
%\begin{minipage}[t]{0.48\textwidth}
%	\begin{picture}(0,0)
%		\put(20,-75){\includegraphics[scale=0.35]{pics/Torus.png}}
%	\end{picture}
%\end{minipage}	

\subsection{Die Partikel}
Partikel sind zufällig generiert mmmkayyy?

\subsection{Das Zentrum}
Das Zentrum ist wie die Sonne, yo

\subsection{Ablauf}
Sachen werden simuliert, verstehste?

\subsection{Visualisierung}
Wir haben auch nen Visualizer, geiler scheiss



\section{Implementation}
\subsection{Sequentieller Algorithmus}
Da läuft das so ab

\subsection{Parallelisierungsschema}
Und hier läuft es so ab

\begin{minted}{C++}
#include <Heidmann.h>
std::cout << "lol" << std::endl;
\end{minted}
%\quad \\


\subsection{Visualizer}
Der Visualizer funktioniert nämlich so

\section{Performance}
Performance? Was ist das?

\section{Probleme}
FOlgende probleme hatten wir lol

\section{Fazit und Ausblick}
\subsection{Fazit}
Das haben wir gelernt
\subsection{Ausblick}
locker machen wir noch mehr Dinge (lolol)

%\pagebreak
%\nocite{*}
