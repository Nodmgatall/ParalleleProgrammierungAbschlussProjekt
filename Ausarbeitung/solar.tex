\title{Simulation eines Sonnensystems}
%TODO :
\subtitle{Praktikumsbericht}

\author{Oliver Heidmann, Tronje Krabbe}

\institute{Universität Hamburg \\ 
  Fakultät für Mathematik, Informatik und Naturwissenschaften \\
  Fachbereich Informatik, DKRZ \\
  Praktikum Paralleles Programmieren 2015}
  
\date{\today}

\maketitle

%\index{personen}{Krabbe, Tronje}
\newpage
\begin{abstract}
%TODO content
\quad \\
Aufgabenstellung: \\
Programmierung einer parallelisierten Applikation mittels MPI. \\ \\
Idee: \\
Simulation von Partikeln in einem Sonnensystem.
%\keywords{Keywords}
\end{abstract}

\tableofcontents
\newpage
\section{Idee}
Die Idee, die unserem Projekt zugrunde liegt, ist relativ simpel zu formulieren:
Wir wollen ein Sonnensystem simulieren. In diesem System sollten die enthaltenen
Objekte realistisch miteinander interagieren. Alle Objekte beinflussen sich durch
die Gravitation und durch physischen Kontakt, also Kollisionen. 

\section{Modellierung}
In unserem System
wird die Bewegung des ganzen Systems nicht berücktsichtigt und somit ist die 
Geschwindigkeit und die Position des sich in der Mitte befindlichen massereichesten
Objektes fixiert.
Kollisionen sind nur teilweise realistisch umgesetzt. Bei einer Kollision werden
zwei Objekte miteinder verschmolzen und es werden keine weiteren kleineren Objekte
erzeugt.
Alle einheiten sind in SI-Einheiten.

%\subsection{Das System}
%Wie bereits erwähnt ist die Welt, in der unsere Simulation stattfindet,
%ein System analog zu einem echten Sonnensystem. Es ist nicht räumlich begrenzt,
%und enthält eine beliebige Anzahl Partikel. Kollidieren zwei Partikel, so wird aus ihnen
%ein neues Partikel errechnet, die Population schrumpft also nur; es werden keine neuen
%hinzugefügt.

%\begin{minipage}[t]{0.48\textwidth}
%    Dreidimensionaler raum mkay?
%\end{minipage}
%\begin{minipage}[t]{0.48\textwidth}
%	\begin{picture}(0,0)
%		\put(20,-75){\includegraphics[scale=0.35]{pics/Torus.png}}
%	\end{picture}
%\end{minipage}	

\subsection{Die Partikel}
Partikel haben folgende Eigenschaften:
\begin{itemize}
    \item Geschwindigkeitsvektor
    \item Position
    \item Masse
    \item Radius
\end{itemize}
Wir gehen von sphärischen Objekten aus. Spezifische Details wie z.B.
Material, Temperatur oder genauere Form werden nicht berücksichtigt.
Partikel werden immer in einer zur Sonne orthogonalen Laufbahn generiert
um die Anzahl von Objekten, die eine stabile Umlaufbahn erreichen können,
zu steigern. Die Dichte von Objekten wird nicht als Datensatz gespeichert,
sondern nur bei Bedarf errechnet, um die Datenmenge zu reduzieren.

\subsection{Das Zentrum}
Im Zentrum unseres Systems befindet sich ein Partikel mit einer besonderen
Eigenschaft. Es ist um ein Vielfaches schwerer als die zufällig generierten Objekte.
Die Existenz dieses Zentralpartikels bewirkt einerseits eine übersichtliche Simulation,
da sich die meisten Partikel irgendwann auf einem Orbit um die `Sonne' finden,
und bringt die Simulation näher an die tatsächlichen Zustände in unserem Universum.

\subsection{Ablauf}
Der Ablauf der Simulation ansich ist relativ simpel; die Partikel werden sortiert,
auf alle die Gravitation appliziert; es wird auf Kollisionen geprüft und dann werden
die Objekte bewegt. Dieser loop wird für die spezifizierte Anzahl Iterationen
fortgeführt.

Die parallelisierte Version hat eine andere Reinfolge.
Hier haben wir die Kollisionsabfrage vor den Bewegungsbefehl und die Errechnung
der neuen Geschwindigkeiten verlegt. Dies geschah um den Datentransfer zwischen
den einzelnen Prozessen zu veringern.

\subsection{Visualisierung}
Zum Zweck der Visualizierung werden für jede Iteration und für jedes Partikel seine
Eigenschaften gespeichert. Diese Daten können dann von einem separaten Programm,
unserem Visualizer, eingelesen und die Simulation damit dargestellt werden.


\section{Implementation}
\subsection{Allgemein}
Den Kern der Simulation bildet die Klasse `Particle':
\begin{minted}{C++}
class Particle
{
    private:
        std::vector<Vec3<double> > m_velocity_vectors;
        std::vector<Vec3<double> > m_positions;
        std::vector<double> m_masses;
        std::vector<double> m_radiuses;
        std::vector<unsigned long> m_ids;
        unsigned long m_number_of_particles;

    /* [...] */
}
\end{minted}

\subsection{Parallelisierungsschema}
In unserer Simulation übernimmt Prozess 0 immer die verwaltung der daten und somit das Verteilen
und Sammeln. Hinzu kommen das Speichern der Simulationsdaten und die Ausgabe des Fortschritts.
Alle anderen Prozesse empfangen die Daten, bearbeiten sie, und senden dann die
durch die Kollisionsabfrage geänderte Länge des Datensatzes an Prozess 0.

\newpage

\subsubsection{Funktionsweise:}
\ \\
ALLE:
\begin{itemize}
   \item Variableninitialisierung am Anfang
   \item Erhöhung des privaten Zählers am Ende
\end{itemize}
PRO 0:
\begin{itemize}
    \item Prozess 0 errechnet die Anteile an den Datenvektoren des Systems
    \item sortiert die Vektoren nach der Entfernung zum Mittelpunkt 
    \item broadcastet die Objekt-Anzahl
    \item broadcastet nacheinander die Datenvektoren
    \item erhält dann in richtiger Reihenfolge die neu berechneten Daten und fügt sie wieder zusammen
    \item überscheibt seine alten Daten mit den neuen
    \item speichert sie
\end{itemize}
PRO 1 - n
\begin{itemize}
    \item empfängt die Broadcasts von 0
    \item errechnet seine Start- und End-Positionen in den Vektoren
    \item berechnet Kollisionen
    \item ändert die Länge der zu bearbeitenden Daten
    \item berechnet die neuen Geschwindigkeiten
    \item berechnet die neuen Positionen
    \item sendet neue Positionen
    \item sendet neue Geschwindigkeiten
\end{itemize}



\subsection{Visualizer}
Der Visualizer visualisiert eine fertige Simulation anhand der generierten Daten.
Es handelt sich dabei um ein separates Programm, ebenfalls in C++ geschrieben,
mit Hilfe von SDL2.

Integriert ist eine provisorische Konsole und mehrere Hotkeys für die einfache Benutzung.

\section{Performance}
ayy lmao

\section{Probleme}
Ein Problem bei der Parallelisierung war die Datenaufteilung. Da Gravitation einen effektiv
unendlichen Einflussradius hat, muss jeder Prozess die gesamten, aktuellen Daten aller
Partikel kennen.

Besonders die Kollisionsabfrage benoetigt alle Daten da bei einer Kollision der Komplette
Datensatz fuer das Objekt geloescht werden muss.

\section{Fazit und Ausblick}
\subsection{Fazit}
Das haben wir gelernt
\subsection{Ausblick}
Für die Zukunft des Projektes sind folgende Features geplant:
Simulation:
\begin{itemize}
    \item verbesserte Kollision die mehrere neue Objekte generiert
    \item verbesserte Performance
    \item neu strukturiertes Senden und Empfangen
    \item Erweiterung des Sendens für Datensätze die die maximale Bufferlänge von MPI überschreiten
    \item performantere Kollisionsabfrage
    \item physikalisch korrekte Errechnung von Kollisionen und den Richtungsvektoren der daraus resultierenden neuen Objekte
    \item zusätzliche Daten z.B. Durchschnittstemperatur
    \item genauere Angaben, welche Daten gespeichert werden und welche nicht
    \item Errechnung und Speicherung für Kollsionsdaten z.B. Impuls, kinetische Energie, Geschwindigkeit bei Einschlag etc.
    \item Berechnung der Roche-Grenze und die Zerstörung von Objekten durch diese Grenze.
\end{itemize}
Visualizer:
\begin{itemize}
    \item Laufbahnen als lienen darstellen
    \item Lienen der Laufbahnen auf Wunsch der Geschwindikeit entsprechend
    \item einfärben
    \item Temperaturberechnung
    \item Objekte bei einem gewissen Zoomfaktor als Kreise Darstellen
    \item Visualizer in 3D, oder wenigstens rotierbar in 2 Achsen
\end{itemize}

\section{Zahlen}
\begin{itemize}
    \item ~2700 Zeilen Code
    \item 163 Commits bis jetzt
    \item $>$20 Gb generierte Simulations-Daten
\end{itemize}

%\pagebreak
%\nocite{*}
